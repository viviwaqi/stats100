% Options for packages loaded elsewhere
\PassOptionsToPackage{unicode}{hyperref}
\PassOptionsToPackage{hyphens}{url}
%
\documentclass[
]{article}
\usepackage{amsmath,amssymb}
\usepackage{lmodern}
\usepackage{iftex}
\ifPDFTeX
  \usepackage[T1]{fontenc}
  \usepackage[utf8]{inputenc}
  \usepackage{textcomp} % provide euro and other symbols
\else % if luatex or xetex
  \usepackage{unicode-math}
  \defaultfontfeatures{Scale=MatchLowercase}
  \defaultfontfeatures[\rmfamily]{Ligatures=TeX,Scale=1}
\fi
% Use upquote if available, for straight quotes in verbatim environments
\IfFileExists{upquote.sty}{\usepackage{upquote}}{}
\IfFileExists{microtype.sty}{% use microtype if available
  \usepackage[]{microtype}
  \UseMicrotypeSet[protrusion]{basicmath} % disable protrusion for tt fonts
}{}
\makeatletter
\@ifundefined{KOMAClassName}{% if non-KOMA class
  \IfFileExists{parskip.sty}{%
    \usepackage{parskip}
  }{% else
    \setlength{\parindent}{0pt}
    \setlength{\parskip}{6pt plus 2pt minus 1pt}}
}{% if KOMA class
  \KOMAoptions{parskip=half}}
\makeatother
\usepackage{xcolor}
\usepackage[margin=1in]{geometry}
\usepackage{color}
\usepackage{fancyvrb}
\newcommand{\VerbBar}{|}
\newcommand{\VERB}{\Verb[commandchars=\\\{\}]}
\DefineVerbatimEnvironment{Highlighting}{Verbatim}{commandchars=\\\{\}}
% Add ',fontsize=\small' for more characters per line
\usepackage{framed}
\definecolor{shadecolor}{RGB}{248,248,248}
\newenvironment{Shaded}{\begin{snugshade}}{\end{snugshade}}
\newcommand{\AlertTok}[1]{\textcolor[rgb]{0.94,0.16,0.16}{#1}}
\newcommand{\AnnotationTok}[1]{\textcolor[rgb]{0.56,0.35,0.01}{\textbf{\textit{#1}}}}
\newcommand{\AttributeTok}[1]{\textcolor[rgb]{0.77,0.63,0.00}{#1}}
\newcommand{\BaseNTok}[1]{\textcolor[rgb]{0.00,0.00,0.81}{#1}}
\newcommand{\BuiltInTok}[1]{#1}
\newcommand{\CharTok}[1]{\textcolor[rgb]{0.31,0.60,0.02}{#1}}
\newcommand{\CommentTok}[1]{\textcolor[rgb]{0.56,0.35,0.01}{\textit{#1}}}
\newcommand{\CommentVarTok}[1]{\textcolor[rgb]{0.56,0.35,0.01}{\textbf{\textit{#1}}}}
\newcommand{\ConstantTok}[1]{\textcolor[rgb]{0.00,0.00,0.00}{#1}}
\newcommand{\ControlFlowTok}[1]{\textcolor[rgb]{0.13,0.29,0.53}{\textbf{#1}}}
\newcommand{\DataTypeTok}[1]{\textcolor[rgb]{0.13,0.29,0.53}{#1}}
\newcommand{\DecValTok}[1]{\textcolor[rgb]{0.00,0.00,0.81}{#1}}
\newcommand{\DocumentationTok}[1]{\textcolor[rgb]{0.56,0.35,0.01}{\textbf{\textit{#1}}}}
\newcommand{\ErrorTok}[1]{\textcolor[rgb]{0.64,0.00,0.00}{\textbf{#1}}}
\newcommand{\ExtensionTok}[1]{#1}
\newcommand{\FloatTok}[1]{\textcolor[rgb]{0.00,0.00,0.81}{#1}}
\newcommand{\FunctionTok}[1]{\textcolor[rgb]{0.00,0.00,0.00}{#1}}
\newcommand{\ImportTok}[1]{#1}
\newcommand{\InformationTok}[1]{\textcolor[rgb]{0.56,0.35,0.01}{\textbf{\textit{#1}}}}
\newcommand{\KeywordTok}[1]{\textcolor[rgb]{0.13,0.29,0.53}{\textbf{#1}}}
\newcommand{\NormalTok}[1]{#1}
\newcommand{\OperatorTok}[1]{\textcolor[rgb]{0.81,0.36,0.00}{\textbf{#1}}}
\newcommand{\OtherTok}[1]{\textcolor[rgb]{0.56,0.35,0.01}{#1}}
\newcommand{\PreprocessorTok}[1]{\textcolor[rgb]{0.56,0.35,0.01}{\textit{#1}}}
\newcommand{\RegionMarkerTok}[1]{#1}
\newcommand{\SpecialCharTok}[1]{\textcolor[rgb]{0.00,0.00,0.00}{#1}}
\newcommand{\SpecialStringTok}[1]{\textcolor[rgb]{0.31,0.60,0.02}{#1}}
\newcommand{\StringTok}[1]{\textcolor[rgb]{0.31,0.60,0.02}{#1}}
\newcommand{\VariableTok}[1]{\textcolor[rgb]{0.00,0.00,0.00}{#1}}
\newcommand{\VerbatimStringTok}[1]{\textcolor[rgb]{0.31,0.60,0.02}{#1}}
\newcommand{\WarningTok}[1]{\textcolor[rgb]{0.56,0.35,0.01}{\textbf{\textit{#1}}}}
\usepackage{graphicx}
\makeatletter
\def\maxwidth{\ifdim\Gin@nat@width>\linewidth\linewidth\else\Gin@nat@width\fi}
\def\maxheight{\ifdim\Gin@nat@height>\textheight\textheight\else\Gin@nat@height\fi}
\makeatother
% Scale images if necessary, so that they will not overflow the page
% margins by default, and it is still possible to overwrite the defaults
% using explicit options in \includegraphics[width, height, ...]{}
\setkeys{Gin}{width=\maxwidth,height=\maxheight,keepaspectratio}
% Set default figure placement to htbp
\makeatletter
\def\fps@figure{htbp}
\makeatother
\setlength{\emergencystretch}{3em} % prevent overfull lines
\providecommand{\tightlist}{%
  \setlength{\itemsep}{0pt}\setlength{\parskip}{0pt}}
\setcounter{secnumdepth}{-\maxdimen} % remove section numbering
\ifLuaTeX
  \usepackage{selnolig}  % disable illegal ligatures
\fi
\IfFileExists{bookmark.sty}{\usepackage{bookmark}}{\usepackage{hyperref}}
\IfFileExists{xurl.sty}{\usepackage{xurl}}{} % add URL line breaks if available
\urlstyle{same} % disable monospaced font for URLs
\hypersetup{
  pdftitle={stat hw 1},
  hidelinks,
  pdfcreator={LaTeX via pandoc}}

\title{stat hw 1}
\author{}
\date{\vspace{-2.5em}2022-09-26}

\begin{document}
\maketitle

\begin{Shaded}
\begin{Highlighting}[]
\FunctionTok{library}\NormalTok{(tidyverse)}
\end{Highlighting}
\end{Shaded}

\begin{verbatim}
## -- Attaching packages --------------------------------------- tidyverse 1.3.2 --
## v ggplot2 3.3.6      v purrr   0.3.4 
## v tibble  3.1.8      v dplyr   1.0.10
## v tidyr   1.2.1      v stringr 1.4.1 
## v readr   2.1.2      v forcats 0.5.2 
## -- Conflicts ------------------------------------------ tidyverse_conflicts() --
## x dplyr::filter() masks stats::filter()
## x dplyr::lag()    masks stats::lag()
\end{verbatim}

\begin{Shaded}
\begin{Highlighting}[]
\NormalTok{pain}\OtherTok{\textless{}{-}} \FunctionTok{read.csv}\NormalTok{(}\StringTok{"pain.csv"}\NormalTok{)}
\NormalTok{score}\OtherTok{\textless{}{-}}\NormalTok{ pain}\SpecialCharTok{$}\NormalTok{Score}
\NormalTok{haircolor}\OtherTok{\textless{}{-}}\NormalTok{ pain}\SpecialCharTok{$}\NormalTok{HairColor}
\end{Highlighting}
\end{Shaded}

\begin{Shaded}
\begin{Highlighting}[]
\FunctionTok{mean}\NormalTok{(score)}
\end{Highlighting}
\end{Shaded}

\begin{verbatim}
## [1] 47.84211
\end{verbatim}

\begin{Shaded}
\begin{Highlighting}[]
\FunctionTok{sd}\NormalTok{(score)}
\end{Highlighting}
\end{Shaded}

\begin{verbatim}
## [1] 11.4565
\end{verbatim}

\begin{Shaded}
\begin{Highlighting}[]
\NormalTok{db}\OtherTok{\textless{}{-}} \FunctionTok{filter}\NormalTok{(pain, haircolor }\SpecialCharTok{==} \StringTok{"DarkBrunette"}\NormalTok{) }
\NormalTok{db}
\end{Highlighting}
\end{Shaded}

\begin{verbatim}
##      HairColor Score
## 1 DarkBrunette    32
## 2 DarkBrunette    39
## 3 DarkBrunette    51
## 4 DarkBrunette    30
## 5 DarkBrunette    35
\end{verbatim}

\begin{Shaded}
\begin{Highlighting}[]
\FunctionTok{mean}\NormalTok{(db}\SpecialCharTok{$}\NormalTok{Score)}
\end{Highlighting}
\end{Shaded}

\begin{verbatim}
## [1] 37.4
\end{verbatim}

\begin{Shaded}
\begin{Highlighting}[]
\NormalTok{lb}\OtherTok{\textless{}{-}} \FunctionTok{filter}\NormalTok{(pain, haircolor }\SpecialCharTok{==} \StringTok{"LightBrunette"}\NormalTok{)}
\NormalTok{lb}
\end{Highlighting}
\end{Shaded}

\begin{verbatim}
##       HairColor Score
## 1 LightBrunette    42
## 2 LightBrunette    50
## 3 LightBrunette    41
## 4 LightBrunette    37
\end{verbatim}

\begin{Shaded}
\begin{Highlighting}[]
\FunctionTok{sd}\NormalTok{(lb}\SpecialCharTok{$}\NormalTok{Score)}
\end{Highlighting}
\end{Shaded}

\begin{verbatim}
## [1] 5.446712
\end{verbatim}

\begin{Shaded}
\begin{Highlighting}[]
\FunctionTok{count}\NormalTok{(lb)}
\end{Highlighting}
\end{Shaded}

\begin{verbatim}
##   n
## 1 4
\end{verbatim}

\begin{Shaded}
\begin{Highlighting}[]
\NormalTok{pain }\SpecialCharTok{\%\textgreater{}\%}
  \FunctionTok{group\_by}\NormalTok{(HairColor) }\SpecialCharTok{\%\textgreater{}\%}
  \FunctionTok{summarise}\NormalTok{(HairColor, }\AttributeTok{mean=}\FunctionTok{mean}\NormalTok{(Score)) }\SpecialCharTok{\%\textgreater{}\%}
  \FunctionTok{arrange}\NormalTok{(}\FunctionTok{desc}\NormalTok{(mean))}
\end{Highlighting}
\end{Shaded}

\begin{verbatim}
## `summarise()` has grouped output by 'HairColor'. You can override using the
## `.groups` argument.
\end{verbatim}

\begin{verbatim}
## # A tibble: 19 x 2
## # Groups:   HairColor [4]
##    HairColor      mean
##    <chr>         <dbl>
##  1 LightBlond     59.2
##  2 LightBlond     59.2
##  3 LightBlond     59.2
##  4 LightBlond     59.2
##  5 LightBlond     59.2
##  6 DarkBlond      51.2
##  7 DarkBlond      51.2
##  8 DarkBlond      51.2
##  9 DarkBlond      51.2
## 10 DarkBlond      51.2
## 11 LightBrunette  42.5
## 12 LightBrunette  42.5
## 13 LightBrunette  42.5
## 14 LightBrunette  42.5
## 15 DarkBrunette   37.4
## 16 DarkBrunette   37.4
## 17 DarkBrunette   37.4
## 18 DarkBrunette   37.4
## 19 DarkBrunette   37.4
\end{verbatim}

\begin{Shaded}
\begin{Highlighting}[]
\NormalTok{pain }\SpecialCharTok{\%\textgreater{}\%}
  \FunctionTok{group\_by}\NormalTok{(HairColor) }\SpecialCharTok{\%\textgreater{}\%}
  \FunctionTok{summarise}\NormalTok{(HairColor, }\AttributeTok{standard\_dev=}\FunctionTok{sd}\NormalTok{(Score)) }\SpecialCharTok{\%\textgreater{}\%}
  \FunctionTok{arrange}\NormalTok{(standard\_dev)}
\end{Highlighting}
\end{Shaded}

\begin{verbatim}
## `summarise()` has grouped output by 'HairColor'. You can override using the
## `.groups` argument.
\end{verbatim}

\begin{verbatim}
## # A tibble: 19 x 2
## # Groups:   HairColor [4]
##    HairColor     standard_dev
##    <chr>                <dbl>
##  1 LightBrunette         5.45
##  2 LightBrunette         5.45
##  3 LightBrunette         5.45
##  4 LightBrunette         5.45
##  5 DarkBrunette          8.32
##  6 DarkBrunette          8.32
##  7 DarkBrunette          8.32
##  8 DarkBrunette          8.32
##  9 DarkBrunette          8.32
## 10 LightBlond            8.53
## 11 LightBlond            8.53
## 12 LightBlond            8.53
## 13 LightBlond            8.53
## 14 LightBlond            8.53
## 15 DarkBlond             9.28
## 16 DarkBlond             9.28
## 17 DarkBlond             9.28
## 18 DarkBlond             9.28
## 19 DarkBlond             9.28
\end{verbatim}

\begin{Shaded}
\begin{Highlighting}[]
\NormalTok{pain }\SpecialCharTok{\%\textgreater{}\%}
  \FunctionTok{group\_by}\NormalTok{(HairColor) }\SpecialCharTok{\%\textgreater{}\%}
  \FunctionTok{count}\NormalTok{()}
\end{Highlighting}
\end{Shaded}

\begin{verbatim}
## # A tibble: 4 x 2
## # Groups:   HairColor [4]
##   HairColor         n
##   <chr>         <int>
## 1 DarkBlond         5
## 2 DarkBrunette      5
## 3 LightBlond        5
## 4 LightBrunette     4
\end{verbatim}

\end{document}
